\documentclass[11pt]{article}
\usepackage{geometry}
\geometry{verbose,letterpaper,tmargin=24mm,bmargin=24mm,lmargin=24mm,rmargin=24mm}
\usepackage{tikz}
\usetikzlibrary{decorations.pathreplacing,hobby,backgrounds,fit,positioning}
\usepackage[thmmarks]{ntheorem}
\usepackage{amssymb}
\usepackage{caption,graphicx,newfloat}
\usepackage[justification=centering]{caption}% or e.g. [format=hang]
\usepackage[linesnumbered]{algorithm2e}

\DeclareCaptionType{InfoBox}

\theoremstyle{plain}
\newtheorem{theorem}{Theorem}
\newtheorem{lemma}[theorem]{Lemma}
\newtheorem{conjecture}[theorem]{Conjecture}
\newtheorem{observation}[theorem]{Observation}
\newtheorem{claim}{Claim}[theorem]
\newtheorem*{notice}{Notice}
\newtheorem{corollary}{Corollary}
\newtheorem*{definition}{Definition}

\makeatletter
\newtheoremstyle{nonumberplainnobrackets}%
  {\item[\theorem@headerfont\hskip\labelsep ##1\theorem@separator]}%
  {\item[\theorem@headerfont\hskip \labelsep ##1\ ##3\theorem@separator]}
\makeatother

\theoremstyle{nonumberplainnobrackets}
\theoremsymbol{\ensuremath{\Box}}
\newtheorem{proof}{Proof}

\theoremsymbol{\ensuremath{\blacksquare}} 
\newtheorem{claimproof}{Proof of claim}
\newtheorem{lemmaproof}{Proof of lemma}



\usepackage{hyperref,amsmath}
\usepackage {tikz}
\usetikzlibrary {positioning,automata,arrows}
\usepackage{caption}
\definecolor {processblue}{cmyk}{0.96,0,0,0}
%opening
\title{Complexity - Exercise 1}
\author{Oren Roth}
\date{\today}

\renewcommand{\thefootnote}{\fnsymbol{footnote}}
\makeatletter
\newcommand\footnoteref[1]{\protected@xdef\@thefnmark{\ref{#1}}\@footnotemark}
\makeatother

\pgfdeclareshape{item}{
	\savedanchor\northeast{\squarecorner{}}
	\savedanchor\southwest{\squarecorner{-}}
	
	\foreach \x in {east,west} \foreach \y in {north,mid,base,south} {
		\inheritanchor[from=rectangle]{\y\space\x}
	}
	\foreach \x in {east,west,north,mid,base,south,center,text} {
		\inheritanchor[from=rectangle]{\x}
	}
	\inheritanchorborder[from=rectangle]
	\inheritbackgroundpath[from=rectangle]
}

\begin{document}

\maketitle
\section*{Question 1}
 ($\Leftarrow$) Let $L^* \in NP-C \cap CO-NP$ and let us show $CO-NP=NP$.
	\begin{claim}
\begin{align*}
L \in NP \Rightarrow \bar{L} \in NP
\end{align*}
\end{claim}
\begin{claimproof}
$L\in NP$ and $L^* \in NP-C$ so there a polynomial reduction function $f$ s.t.:
\begin{align*}
	L \le_p L^*
\end{align*}
Therefore:
\begin{align*}
x\in L \iff f(x)\in L^*
\end{align*}
By the definition of complement languages:
\begin{align*}
x\in \bar{L} \iff x\notin L \iff f(x)\notin L^* \iff f(x)\in  \bar{L^*}
\end{align*}
So we conclude that $f$ is a polynomial reduction for:
\begin{align*}
\bar{L} \le_p \bar{L^*}
\end{align*}
And we know $L^*\in CO-NP$ so $\bar{L^*} \in NP$ and thus $\bar{L}\in NP$ 
\end{claimproof}
After we showed the proof of the claim we will use it to show $CO-NP=NP$.
Let $L\in NP$, by the claim:
\begin{align*}
\bar{L} \in NP \Rightarrow L \in CO-NP
\end{align*}
Let $L\in CO-NP$, so we know $\bar{L}\in NP$ and by the claim:
\begin{align*}
L \in NP
\end{align*}
We conclude $NP=CO-NP$.\\
\\
($\Rightarrow$) Let $NP=CO-NP$, we know $SAT\in NP-C$ so,
\begin{align*}
SAT \in NP \Rightarrow SAT \in CO-NP 
\end{align*} 
and we showed there is language in $NP-C \cap CO-NP$

\section*{Question 2}
\begin{enumerate}
	\item 
	Define:
	\begin{align*}
		Id: \Sigma^* \rightarrow \Sigma^*\\
		\forall x\in \Sigma^*,\quad Id(x) = x
	\end{align*}
	Let $A\subseteq \Sigma^*$ then:
	
	\begin{align*}
	x\in A \iff 
	 Id(x) = x \in A
	\end{align*}
	And thus $Id$ is reduction from $A$ to $A$. Clearly $Id$ runs $O(1)$.
	\item 
It's false.\\
	We know $L_{halt} \in NP-Hard\setminus NP$ so for any $L\in NP$:
		\begin{align*}
	L \le_p L_{halt}
	\end{align*}
	but it if:
		\begin{align*}
	L_{halt} \le_p L
	\end{align*}
	then $L_{halt} \in NP$ and contradiction. And hence,
	\begin{align*}
	L_{halt} \nleq_p L
	\end{align*}
	\item 
	It's false.\\
	$SAT,SAT-3\in NP-Complete$ and thus:
	\begin{align*}
		SAT \le_p SAT-3\\
				SAT-3 \le_p SAT
	\end{align*}
	But $SAT \ne SAT-3$.
	\item Let $A \le_p B$ by function $f$, hence:
	\begin{align*}
	x\in A \iff f(x)\in B
	\end{align*}
	By the definition of complement languages:
	\begin{align*}
	x\in \bar{A} \iff x\notin A \iff f(x)\notin B \iff f(x)\in  \bar{B}
	\end{align*}
	So we conclude that $f$ is a polynomial reduction for:
	\begin{align*}
	\bar{A} \le_p \bar{B}
	\end{align*}
\end{enumerate}

\section*{Question 3}
\begin{claim}
	For any $\phi = (x_1 \vee x_2 \vee \ldots \vee x_k)$ with $k\ge 4$ there is CNF $\phi'$ with 2 clauses in length smaller than $k$ each, defined over $x_1,\ldots x_k$ and $y$(new variable). With the following property:
	\begin{align*}
	\phi(X) = true \iff \exists y\;s.t\;\phi'(X,y) = true
	\end{align*}
\end{claim}
\begin{claimproof}
	Let $\phi = (x_1 \vee x_2 \vee \ldots \vee x_k)$ with $k\ge 4$, we define:
	\begin{align*}
	\phi' = (x_1 \vee x_2 \vee \ldots \vee x_{k-2} \vee y) \wedge (x_{k-1} \vee x_k  \vee \neg y)
	\end{align*}	
If there is $X$ assignment for $x_1,\ldots,x_k$ s.t. $\phi(X)= true$, if one of $x_{k-1},x_k$ got true we set $y=true$ and $\phi'(X,y) = true$ (the first clause is true due to $y$ and the second due to  $x_{k-1},x_k$). And otherwise if both $x_{k-1},x_k$ are not true it has to be the case that one of $x_1,\ldots x_{k-2}$ literals got true and setting $y=false$ gain $\phi'(X,y) = true$.
On the other hand if $\phi'(X,y) = true$ for some $y$, if $y=true$ than it has to be the case than one of  $x_{k-1},x_k$ are true and thus  $\phi(X)=true$ and else if $y=false$  it has to be the case than one of  $x_1,\ldots x_{k-2}$ literals got true and still $\phi(X)=true$.
\end{claimproof}
\begin{claim}
	For any $\phi=\bigwedge_{i=1}^n C_i$, an CNF for $SAT$ there is CNF $\phi'=\bigwedge_{i=1}^m C'_i$ for $3-SAT$ with $poly(n)$ clauses on the same variables. $\phi,\phi'$ will agree between them. Moreover $\phi'$ can be build from $\phi$ in polynomial time.
\end{claim}
\begin{claimproof}
	Let $\phi = \bigwedge_{i=1}^n C_i$, an CNF for $SAT$, define $k = max\{|C_i|\}$ we will show by induction on $k$ that we can create $\phi'$.\\
	\textbf{Basis:}
	\begin{enumerate}
		\item \textbf{k=1:} instead of $C_i= x_1$ we will create:
		\begin{align*}
			C_i' = (x_1 \vee y_i \vee z_i) \wedge  (x_1 \vee \neg y_i \vee z_i) \wedge  (x_1 \vee  y_i \vee \neg z_i) \wedge  (x_1 \vee \neg y_i \vee \neg z_i) 
		\end{align*}
		With two new variables ($y_i,z_i$), and no matter what is the assignment of $y_i,z_i$ it will be one clause in $C_i'$ that $x_1$ will need to have true. On the other side if $x_1$ is true than all the clauses in $C_i'$ are true.
		\item \textbf{k=2:} Clauses of length 1 we will handle as before, otherwise, instead of $C_i= (x_1\vee x_2) $ we will create:
			\begin{align*}
		C_i' = (x_1 \vee x_2 \vee y_i) \wedge  (x_1 \vee x_2 \vee \neg y_i) 
		\end{align*}
		With new variables ($y_i$), again no matter what is the assignment of $y_i$ it will be one clause in $C_i'$ that $x_1,x_2$ will need to have true. On the other side if $x_1,x_2$ is true than all the clauses in $C_i'$ are true.
		\item \textbf{k=3:} Clauses of length $<$ 3 we will handle as before, otherwise, we will keep $C_i' = C_i$ as it has 3 literals in each clause.	
	\end{enumerate}
Assume that for any $k$ with $k < n$ we can define $\phi'$ and let be $\phi$ with $k = max\{|C_i|\}=n\ge4$.  For any $C_i$ with length $n$ by the previous lemma we know we can create two clauses with length $n-1$ that has the same truth values and know we have new formula with all clauses of length  $<n$ and by the assumption we can handle those.\\
Notice that for each clause with $m$ literals we create at most $m$ new clauses, and so this process is polynomial in $n$.
\end{claimproof}
It's clear $3-SAT\in NP$ as given $\phi$ an assignment for $\phi$ can be the proof for showing whereas $\phi \in 3-SAT$ (As there is a valid assignment for $\phi$ iff $\phi \in 3-SAT$).  To show $3-SAT\in NP-hard$ we will just show polynomial reduction between $SAT$ to $3-SAT$, because $SAT\in NP-Complete$ it will admit immediately that $3-SAT\in NP-hard$.\\
Given, $\phi=\bigwedge_{i=1}^n C_i$, an CNF for $SAT$ we will build in polynomial time $\phi'$ for $3-SAT$ by the lemma, and we got $\phi \in SAT \iff \phi' \in 3-SAT$.

\section*{Question 4}
We will use an array $Arr$ with size $m$.\\

$H(m):$\\
\begin{algorithm}[H]
	
	Arr[1] $\leftarrow$ 1\\
	\For{M=2 to m}{
		i=1\\
		\eIf{i $<$ loglog(m)}{ 
			\For{$x \in \{0,1\}^{log(M)}$}{
				run $M_i(x)$ for $i|x|^i$ steps and put the result in $res1$ 
				(e.g $res1 = true$ iff $M_i(x)$ stop and accept after at most $i|x|^i$ steps)\\
				$l \leftarrow$ the index of the rightmost '0' in $x$\\
				$x_{start} \leftarrow x[0,l]$\\
				$x_{end} \leftarrow x[l+1,|x|]$\\
				\If{$|x_{end}|\ne |x_{start}|^{Arr[|x_{start}|]}$}{
					$res2  \leftarrow$ false
					}
				\ElseIf{$x_{start}$ is not CNF clause}{
					$res2  \leftarrow$ false
				}
				\Else{
					brute force to find satisfing assignment for $x_{start}$ and set $res2 = true$ iff one found.
				}
				\If{$res1 \ne res2$}{
					$i\leftarrow i +1$\\
					goto 4
				}
			}
			Arr[M] $\leftarrow$ i
		}{
		Arr[M] $\leftarrow$ loglog(M)
		}
	}
	\Return Arr[m]
\end{algorithm}
\subsection*{Explanation for the run time:}
\begin{itemize}
\item The loop in line 2 runs in linear time in $m$.
\item We go back to line 4 at most $loglog(m)$ times.
\item The loop in line 5 runs at most $2^{log(M)}<2^{log(m)} = m$ times.
\item Line 6 runs in $i|x|^i < loglog(m)log(m)^{loglog(m)}=O(m)$ steps*.
\item Line 7-12 run in polynomial time in $x$ which is polynomial in $m$. 
\item The CNF clause in Line 13  has at most $|x| = log(M) < log(m)$ variables and thus need to check at most $2^{log(m)} = m$ assignments.
\end{itemize}
[*] \includegraphics[scale=.6]{screenshot001}\\ \\
\textbf{Summary} - Each of the lines run in polynomial time in $m$, and because polynom of polynom is still polynom also finite set of cascading loops are still polynom in $m$.
\end{document}
