\documentclass[11pt]{article}
\usepackage{geometry}
\geometry{verbose,letterpaper,tmargin=24mm,bmargin=24mm,lmargin=24mm,rmargin=24mm}
\usepackage{tikz}
\usetikzlibrary{decorations.pathreplacing,hobby,backgrounds,fit,positioning}
\usepackage[thmmarks]{ntheorem}
\usepackage{amssymb}
\usepackage{caption,graphicx,newfloat}
\usepackage[justification=centering]{caption}% or e.g. [format=hang]

\usepackage{algorithm}
\usepackage[noend]{algpseudocode}


\makeatletter
\def\BState{\State\hskip-\ALG@thistlm}
\makeatother

\DeclareCaptionType{InfoBox}

\theoremstyle{plain}
\newtheorem{theorem}{Theorem}
\newtheorem{lemma}[theorem]{Lemma}
\newtheorem{conjecture}[theorem]{Conjecture}
\newtheorem{observation}[theorem]{Observation}
\newtheorem{claim}{Claim}[theorem]
\newtheorem*{notice}{Notice}
\newtheorem{corollary}{Corollary}
\newtheorem*{definition}{Definition}

\makeatletter
\newtheoremstyle{nonumberplainnobrackets}%
  {\item[\theorem@headerfont\hskip\labelsep ##1\theorem@separator]}%
  {\item[\theorem@headerfont\hskip \labelsep ##1\ ##3\theorem@separator]}
\makeatother

\theoremstyle{nonumberplainnobrackets}
\theoremsymbol{\ensuremath{\Box}}
\newtheorem{proof}{Proof}

\theoremsymbol{\ensuremath{\blacksquare}} 
\newtheorem{claimproof}{Proof of claim}
\newtheorem{lemmaproof}{Proof of lemma}



\usepackage{hyperref,amsmath}
\usepackage {tikz}
\usetikzlibrary {positioning,automata,arrows}
\usepackage{caption}
\definecolor {processblue}{cmyk}{0.96,0,0,0}
%opening
\title{Complexity - Exercise 3}
\author{Oren Roth - 200701068}
\date{\today}

\renewcommand{\thefootnote}{\fnsymbol{footnote}}
\makeatletter
\newcommand\footnoteref[1]{\protected@xdef\@thefnmark{\ref{#1}}\@footnotemark}
\makeatother

\pgfdeclareshape{item}{
	\savedanchor\northeast{\squarecorner{}}
	\savedanchor\southwest{\squarecorner{-}}
	
	\foreach \x in {east,west} \foreach \y in {north,mid,base,south} {
		\inheritanchor[from=rectangle]{\y\space\x}
	}
	\foreach \x in {east,west,north,mid,base,south,center,text} {
		\inheritanchor[from=rectangle]{\x}
	}
	\inheritanchorborder[from=rectangle]
	\inheritbackgroundpath[from=rectangle]
}

\begin{document}

\maketitle
\section*{Question 1}
Assume by contradiction:
\begin{align}\label{np}
DSPACE(n^3) = NP
\end{align}
Let $S\in DSPACE(n^9) \setminus DSPACE(n^3)$, we assure it exists by space hierarchy theorem, let $M_S$ be the TM that determine $S$ in $n^9$ time, denote a new language:
\begin{align*}
S' = \{ 1^{|x|^3} 0x : \forall x\in S   \}
\end{align*}
Now we can have new TM, $M_{S'}$, which given input $w$ will:
\begin{enumerate}
	\item  Count in $i$ and delete the leading 1's until the first zero, delete the first zero as well - denote by $x$ the word that left on the tape
	\item  Assure $|x|^3 = i$, if not reject
	\item run $M_S$ on the $x$ and return as $M_S$
\end{enumerate}
$M_{S'}$ will determine $S'$:
\begin{align*}
w\in S' \iff  w =  1^{|x|^3} 0x \wedge  x\in S  \iff M_{S'} \text{ on }w \text{ will pass step 2 without rejects and will accept in step 3} \\ \iff M_{S'} \;accepts  \; w
\end{align*}
The time $M_{S'}$ runs steps 1,2 in linear time with respect to the input size, we pass step 2 only if we run $M_{S'}$ on input from the form $w=1^{|x|^3} 0x$. Step 3 run $M_S$ on $x$ which takes $|x|^9$ time, which is smaller than $|w|^3$ time (As $|w| \ge |x|^3$). Concluding $M_{S'}$ runs in $n^3$ time.
\begin{align*}
S' \in  DSPACE(n^3)
\end{align*}
By (\ref{np}) we know $S' \in NP$, and hence there is $M'$, TM for which 
\begin{align*}
x\in S' \iff \exists y\;s.t.\; M;(x,y) \;accepts
\end{align*}
Given $M'$, we can show that $S\in NP$ by showing the following TM $M$. On input $(x,y)$, $M$ add leading $|x|^3$ 1's to $x$ and runs and return as $M'(1^{|x|^3} 0x,y)$, clearly $M$ runs in polynomial time, now we have:
\begin{align*}
x\in S \iff 1^{|x|^3} 0x\in S' \iff \exists y\;s.t.\; M'(1^{|x|^3} 0x,y) \;accepts \iff  \exists y\;s.t.\; M(x,y) \;accepts
\end{align*}
So we conclude $S\in NP$ and by  (\ref{np}) we know $S \in  DSPACE(n^3)$ which is contradiction to the definition of $S$.
\section*{Question 2}
\subsection*{a.} Let $A\in PP$ and let $M$ be the probabilistic TM that determine it as promise by the definition of $PP$. $M$ is polynomial, and let $p(n)$ be the upper bound of steps $M$ takes for input of size $n$. W.L.O.G assume $M$ runs exactly $p(n)$ steps for each input and for each random calculation route. (If $M$ is not like that we can look at another TM which acts like $M$ and for each calculation route if it's not exceed $p(n)$ steps than it does dummy steps until it reaches and rejects/accepts).\\
\\
Let us define $M'$, deterministic TM which will determine $A$ in polynomial space. $M'$ will hold two counters, one for counting all the accepting paths of $M$ on given input $x$. The second counter will count all the rejecting paths of $M$ on $x$. Given $x$, $M'$ will simulate $M$ on $x$ but instead of using a random bit to decide whereas using $\delta_1$ or $\delta_2$ it will go over all possible $2^{p(|x|)}$ options (We assume each calculation route has exactly $p(n)$ steps and hence $2^{p(|x|)}$ calculation routes). Namely, $M'$ will go over all $r\in \{0,1\}^{p(|x|)}$, for each $r$, it will simulate $M$ where in each step $i$, $M$ will take $\delta_{r[i]}$. In the end of each simulation $M'$ will add 1 to the corresponding counter (depends if it accepts or rejects in the end). $M'$ accepts if and only if the first counter is strictly bigger than the second counter.\\
\\
We notice that number of accepting routes of $M$ is corresponding exactly with the probability that $M$ accepts $x$. So we have:
\begin{align*}
M'\;accepts\; x\iff M \text{ has more accepting routes which accepts }x\iff \Pr[M\;accpets \;x]>\frac{1}{2}\\\iff x\in A 
\end{align*} 
$M'$ use two counters of size $\log (2^{p(|x|)})=p(|x|)$ and tape to go over all the possible $r$'s which also needs $p(|x|)$ bits. To simulate $M$ we need another tape with maximum $p(|x|)$ space as $M$ runs at most $p(|x|)$ steps. In total $M'$ use at most $4\cdot p(|x|)$ space, so we conclude $A\in PSPACE$.
\subsection*{b.} Given $x$, $N$ in the first step will use a coin $r$, identical distribute over $\{0,1\}$. If $r=0$, $N$ will run $M$ on $x$ and return as $M$, otherwise $N$ accepts.
\begin{align*}
\Pr[N\;accepts\;x]=\frac{1}{2}+\frac{1}{2}\cdot \Pr[M\;accepts\;x]
\end{align*}
\subsection*{c.} Given $A\in NP$ let $M$ be non deterministic TM for which:
\begin{align*}
x\in A \iff \text{there is accepting route in the run of } M \text{ on } x
\end{align*}
Let $N$ be the corresponding TM for $M$ as described in the previous clause (2b.). We claim for $N$: 
\begin{align*}
x\in A&\quad \Pr[N\;accepts\;x]>\frac{1}{2}\\
x\notin A&\quad \Pr[N\;accepts\;x]\le\frac{1}{2}
\end{align*}
By question (2b.) we knoe:
\begin{align*}
\Pr[N\;accepts\;x]= \frac{1}{2}+\frac{1}{2}\cdot \Pr[M\;accepts\;x]
\end{align*}
Well if $x\in A$ there is accepting route of $M$ so
\begin{align*}
 \Pr[M\;accepts\;x] > 0
\end{align*}
So we get:
\begin{align*}
\Pr[N\;accepts\;x]= \frac{1}{2}+\frac{1}{2}\cdot \Pr[M\;accepts\;x] >  \frac{1}{2}
\end{align*}
Otherwise if $x\notin A$ there is no accepting route of $M$ so
\begin{align*}
\Pr[M\;accepts\;x] = 0
\end{align*}
And we get:
\begin{align*}
\Pr[N\;accepts\;x]= \frac{1}{2}+\frac{1}{2}\cdot \Pr[M\;accepts\;x] =  \frac{1}{2}
\end{align*}
And we proof our claim, concluding $A\in PP$. And hence:
\begin{align*}
NP\subseteq PP
\end{align*}



\section*{Question 3}
\subsection*{a.}
%Denote by $A$ all the sub graphs $G|_U$:
%\begin{align*}
%A= \{ G|_U : |U|=2\log n  \}
%\end{align*}
%%Notice that $|A| = \frac{2\log n}{n}$.\\
%Denote by $\mathbb{E}[*]$ the expected maximal size IS in $G|_U$ that uniformly selected from $A$.
%We will show that $\mathbb{E}[*]$ is at least $\frac{2k\log n}{n}$. Let $I^*$ be the IS of size $k$ in $G$, divide to cases:
%
%\begin{enumerate}
%	\item $k>2\log n$\\
%	At least $k$ graphs $A' \subseteq A$ are sub graphs of $I^*$, each $G|_U\in A'$ is IS as well of size $2\log n$, $G|_U$ is selected uniformly from $A$, and hence we get:
%	\begin{align*}
%&\mathbb{E}[*] \ge \sum_{G|_U\in A'}(\Pr[\text{choosing $G|_U$}] \cdot |\text{maximal IS in $G|_U$}|) =\\
%&= \sum_{G|_U\in A'}((\frac{2\log n}{n} \cdot (2\log n)) \ge  k \cdot(\frac{2\log n}{n} \cdot (2\log n)) \ge k \cdot(\frac{2\log n}{n} ) 
%	\end{align*}
%	\item $k\le2\log n$\\
%	So we have at least one set $I\in A$ which contains $I^*$, it has maximal IS in size $k$ as well. And for sure $\mathbb{E}[*]$ is at least the size of the maximal size in $I$ times the probability to choose it:
%		\begin{align*}
%	&\mathbb{E}[*] \ge k \cdot(\frac{2\log n}{n} ) 
%	\end{align*}
%\end{enumerate}
Let $U\sim^{U} V$, denote:
\begin{align*}
U = \{ u_1,u_2,\ldots,u_{2\log n}	 \}
\end{align*}
Let $I^*$ be the IS of size $k$ in $G$. We denote $(2\log n)$ random variables, $Y_i$'s, for each $1\le i \le (2 \log n)$:
	\begin{align*}
Y_i = \begin{cases}
1  & \text{if } u_i\in I^* \\
0  & \text{otherwise}
\end{cases} 
\end{align*}
Denote new random variable

\begin{align*}
Y = \sum_{i=1}^{2\log n} Y_i
\end{align*}
Denote by $\mathbb{E}[*]$ the expected maximal size IS in $G|_U$, when $U$ is uniformly drawn from $V$. 
So we having:
\begin{align*}
&\mathbb{E}[*] \ge \mathbb{E}[Y]
\end{align*}
Due to the fact than any choice for all the $u_i$'s get the LHS bigger than the RHS (as on the RHS a choice for $u_i$ with value $Y=r$ says that there is at least IS of size $r$ in $G|_U$, namely all $Y_i$ that equal 1 are correspond to $u_i$'s which are IS of size $r$ and it could be that there is even a bigger IS in $G|_U$).
We need to show:
	\begin{align*}
&\mathbb{E}[*] \ge k \cdot(\frac{2\log n}{n} ) 
\end{align*}
We will show \begin{align*}
& \mathbb{E}[Y] \ge  k \cdot(\frac{2\log n}{n} ) 
\end{align*}
And from transitivity we will gain our goal. 
\begin{align*}
&\mathbb{E}[Y_i] = \frac{k}{n}
\end{align*}
As for each node $u\in V$ there is $k$ over $n$ options to be in $I^*$.
\begin{align*}
\mathbb{E}[Y] = \mathbb{E}[ \sum_{i=1}^{2\log n} Y_i] =  \sum_{i=1}^{2\log n} \mathbb{E}[ Y_i] = 2\log n \cdot \frac{k}{n}
\end{align*}
And we conclude:
	\begin{align*}
&\mathbb{E}[*] \ge k \cdot(\frac{2\log n}{n} ) 
\end{align*}
Denote by $A$ all the sub graphs $G|_U$:
\begin{align*}
A= \{ G|_U : |U|=2\log n  \}
\end{align*}
Now we will show that at least $\alpha$ of the graphs in $A$ have IS of size $\frac{k\log n }{n}$, where $\alpha=\frac{n-2\log n}{n-\log n}$. If by contradiction this is not true, we will have less than $\alpha$ of the graphs in $A$ having maximal IS of size smaller than $\frac{k\log n }{n}$ and for the rest of the graphs have IS of size at most $k$, so we conclude:
	\begin{align*}
&\mathbb{E}[*] < \alpha \cdot \frac{k\log n}{n}  + (1-\alpha)\cdot k = k \cdot(\frac{2\log n}{n} ) 
\end{align*}
But this is contradiction to what we showed before. So we conclude at least $\alpha$ of the graphs in $A$ have IS of size $\frac{k\log n }{n}$.
\begin{align*}
\Pr[\text{maximal IS size in }G|_U < \frac{k\log n }{n}]\le 1-\alpha = \frac{\log n}{n- \log n}
\end{align*}
\subsection*{b.}
For any $n>10$ we have $\alpha >1/2$, means at least an half of the graphs in $A$ have IS of size $\frac{k\log n }{n}$. We will use the following algorithm to solve it for any $n>10$ (for less than that we could just use exhausted search in polynomial time).
\begin{enumerate}
	\item Choose uniformly $U$ subset of size $2\log n$ from $V$.
	\item Go over all the subset of $U$ if one of the is IS of size $\frac{k\log n }{n}$ return it.
	\item Otherwise, return false. 
\end{enumerate}
The algorithm is polynomial, as we have $2^{2\log n} = n^2$ subsets of $U$, so we can go over each on of them, in polynomial time. For each subset we can check its size and check there is no edge between any pair of nodes in $\log ^2 n$ time.\\ \\
If the algorithm return anything but false it's only because it found a IS of size $\frac{k\log n }{n}$. So:
	\begin{align*}
\Pr[\text{The algorithm got wrong answer}] = \text{the algorithm reach line 3}
\end{align*}
Which as we showed in (3a) there is less than half a chance to be in this scenario.
\section*{Question 4}
We will show  $coNP\subseteq IP'$ and $IP'\subseteq coNP$.\\ \\
\underline{$IP'\subseteq coNP$}: Let $A\in coNP$ so we know there is a TM $M$ s.t.
\begin{align*}
x\in A \iff \forall y :\;M(x,y)\; accepts
\end{align*}
Let us show $A\in IP'$ by describing protocol of a prover $P$ and Verifier $V$. \\
Given $x$, $P$ will send to the $V$, $y$. $V$ will run $M(x,y)$ and answer as $M$.\\
\\
If $x\in A$, for any prover $P$, no matter which $y$ it sends, $V$ will accepts with probability 1. (As $M(x,y)$ accepts for all $y$). \\
If $x\notin A$ than by definition there exists $y$ s.t. $M(x,y)$ rejects. So the prover that sends this $y$ will make $P$ accepts with probability 0, which is smaller than one half.  \\
We conclude:
\begin{align*}
A\in IP'
\end{align*}
\\
\underline{$coNP\subseteq IP'$}: Let $A\in IP'$, and let be $V$ the verifier protocol, so we define non-deterministic TM $M$ which will show $A\in coNP$. On given input $x$, $M$ will guess $y=(a,r)$ where $a$ will be a guess of all answers of some prover for all the questions of $V$, and $r$ will be a guess of the coins that $V$ takes. As $IP'$ define both $a,r$ are polynomial in $x$. $M$ will simulate $V$ on answers $a$ with the coins $r$ and accepts iff $V$ accepted.
\\
\\
If $x\in A$, for any prover $P$, $V$ will accepts for all answers with probability 1 $\Rightarrow$ for any guess $y=(a,r)$, $M(x,y)$ accepts.\\
If $x\notin A$, there exists prover s.t $\Pr[V\;accepts \;x]\le \frac{1}{2}$ $\Rightarrow$ for this prover's answers $a$, not all the calculation routes of $V$ are accepting $\Rightarrow$ there exists $y=(a,r)$ s.t. $V(x)$ with coins $r$ and answers $a$ rejects $\Rightarrow$ Exists y s.t. $M(x,y)$ rejects.\\
We conclude:
\begin{align*}
x\in A\iff \forall y:\; M(x,y) \; accepts
\end{align*}
And hence:
\begin{align*}
A\in coNP
\end{align*}
\end{document}




















