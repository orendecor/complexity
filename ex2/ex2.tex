\documentclass[11pt]{article}
\usepackage{geometry}
\geometry{verbose,letterpaper,tmargin=24mm,bmargin=24mm,lmargin=24mm,rmargin=24mm}
\usepackage{tikz}
\usetikzlibrary{decorations.pathreplacing,hobby,backgrounds,fit,positioning}
\usepackage[thmmarks]{ntheorem}
\usepackage{amssymb}
\usepackage{caption,graphicx,newfloat}
\usepackage[justification=centering]{caption}% or e.g. [format=hang]

\usepackage{algorithm}
\usepackage[noend]{algpseudocode}


\makeatletter
\def\BState{\State\hskip-\ALG@thistlm}
\makeatother

\DeclareCaptionType{InfoBox}

\theoremstyle{plain}
\newtheorem{theorem}{Theorem}
\newtheorem{lemma}[theorem]{Lemma}
\newtheorem{conjecture}[theorem]{Conjecture}
\newtheorem{observation}[theorem]{Observation}
\newtheorem{claim}{Claim}[theorem]
\newtheorem*{notice}{Notice}
\newtheorem{corollary}{Corollary}
\newtheorem*{definition}{Definition}

\makeatletter
\newtheoremstyle{nonumberplainnobrackets}%
  {\item[\theorem@headerfont\hskip\labelsep ##1\theorem@separator]}%
  {\item[\theorem@headerfont\hskip \labelsep ##1\ ##3\theorem@separator]}
\makeatother

\theoremstyle{nonumberplainnobrackets}
\theoremsymbol{\ensuremath{\Box}}
\newtheorem{proof}{Proof}

\theoremsymbol{\ensuremath{\blacksquare}} 
\newtheorem{claimproof}{Proof of claim}
\newtheorem{lemmaproof}{Proof of lemma}



\usepackage{hyperref,amsmath}
\usepackage {tikz}
\usetikzlibrary {positioning,automata,arrows}
\usepackage{caption}
\definecolor {processblue}{cmyk}{0.96,0,0,0}
%opening
\title{Complexity - Exercise 2}
\author{Oren Roth}
\date{\today}

\renewcommand{\thefootnote}{\fnsymbol{footnote}}
\makeatletter
\newcommand\footnoteref[1]{\protected@xdef\@thefnmark{\ref{#1}}\@footnotemark}
\makeatother

\pgfdeclareshape{item}{
	\savedanchor\northeast{\squarecorner{}}
	\savedanchor\southwest{\squarecorner{-}}
	
	\foreach \x in {east,west} \foreach \y in {north,mid,base,south} {
		\inheritanchor[from=rectangle]{\y\space\x}
	}
	\foreach \x in {east,west,north,mid,base,south,center,text} {
		\inheritanchor[from=rectangle]{\x}
	}
	\inheritanchorborder[from=rectangle]
	\inheritbackgroundpath[from=rectangle]
}

\begin{document}

\maketitle
\section*{Question 1}
 Let $M$ be the TM which decide $A$ and $p(n)$ a polynomial, which for any input with size $n$, $M$ runs at most $p(n)$ steps with oracle access to inputs with length at most $n-1$.
 We will describe $M'$ that will use only $n*p(n)$ space for any input of size $n$ and decide $A$. This will show $A\in PSPACE$.\\
Given $x\in \Sigma^*$, $M'$ will save $n$ slots with size $p(n)$ each. $M'$ runs as follows:

\begin{itemize}
	\item Copy $x$ to the first slot
	\item run $M$ on the input in slot 1. When oracle request is asked copy the input for the oracle to the next slot and runs $M$ on it. 
	\item Continue the same process when oracle request is asked copy the input for the oracle to the next slot and runs $M$ on it and so on. 
	\item When $M$ answer on input of slot $i>1$, clear slot $i$ and return the answer to $M$ on slot $i-1$ that asked the question to the oracle and continue the run of $M$ on slot $i-1$.
	\item When answer of $M$ appears on slot 1, $M'$ return the same.
\end{itemize}
First notice that any calls for oracle is shrink the input by one, so we will use at most $n$ slots, because each slot is with size $p(n)$, $M'$ uses $poly(n)$ space. Moreover as $M$ need $p(n)$ time for an input in size $n$, it for sure use at most $p(n)$ space for inputs in size $\le n$, in each slot we run on inputs with size $\le$ to $n$ and hence each slot as sufficient space for running $M$.\\  
If all the calls for the oracle are correct $M'$ is just simulating $M$ on slot 1 and hence $L(M')=L(M)$. 
Assume that all the calls for the oracle in input $y$ with $|y|<k$ are simulating correctly by $M'$ and let $y$ be input for the oracle with $|y|=k$, for this we will open new slot and run $M$ on $y$, with every oracle call on inputs in size at most $k-1$ which by the induction assumption we know $M'$ simulate correctly and hence, $M'$ on $y$ will answer exactly as $M$ on $y$.
\section*{Question 2}
We will assume $CO-NP\subseteq NP$, and derive $PH\subseteq NP$ and hence $PH=NP$.\\
\begin{align*}
PH = \bigcup_{i=0}^\infty \Sigma_i^p
\end{align*}
We will show by induction on $i$ that $\Sigma_i^p\subseteq NP$.\\
$i=0$,  $\Sigma_0^p= P\subseteq NP$.\\
$i=1$,  $\Sigma_1^p= NP\subseteq NP$.\\
Assume that for $i$, $2\le i<k$: 
\begin{align*}
\Sigma_i^p\subseteq NP
\end{align*}
We know $NP\subseteq \Sigma_i^p$ and hence $NP=\Sigma_i^p$.\\
This makes $\Pi_i^p = CO-NP$, and because we know $CO-NP\subseteq NP$ we get $\Pi_i^p \subseteq \Sigma_i^p$.
\\
Let us show $\Sigma_k^p\subseteq NP$:
\begin{align*}
	&L \in \Sigma_k^p \iff \exists M\;polynomial\;TM\;s.t:\;\\
	&x\in L \iff \exists y_1\forall y_2 \ldots Q_ky_k \;M(x,y_1,y_2,\ldots,y_k) = 1
\end{align*}
Let us define $L'$:
\begin{align*}
L'= \{ (x,y_1) :\forall y_2 \exists y_3 \ldots Q_ky_k \;M(x,y_1,y_2,\ldots,y_k) =1\}
\end{align*}
So we have:
\begin{align}\label{LL'}
x\in L \iff \exists y_1\forall y_2 \ldots Q_ky_k \;M(x,y_1,y_2,\ldots,y_k) = 1 \iff\\
\forall y_2 \exists y_3 \ldots Q_ky_k \;M(x,y_1,y_2,\ldots,y_k) =1 \iff 
(x,y_1)\in L'
\end{align}
By definition $L'\in \Pi_{k-1}^p$ but we showed for all indexes $i<k$ that $\Pi_i^p \subseteq \Sigma_i^p$, and so we conclude $\Pi_{k-1}^p \subseteq  \Sigma_{k-1}^p$, and:
\begin{align*}
L'\in \Sigma_{k-1}^p
\end{align*}
and hence there exists polynomial $M'$ s.t. 
\begin{align*}
(x,y_1) \in L' \iff \exists y_2  \forall  y_3\ldots Q_ky_k \;M'(x,y_1,y_2,\ldots,y_k) =1
\end{align*}
And moreover:
\begin{align*}
(x,y_1) \in L' \iff \exists y_2  \forall y_3\ldots Q_ky_k \;M'(x,y_1,y_2,\ldots,y_k) =1 \iff \\
\exists y_1,y_2  \forall y_3\ldots Q_ky_k \;M'(x,y_1,y_2,\ldots,y_k) =1 
\end{align*}
Adding with the inequality in (\ref{LL'})  we conclude:
\begin{align*}
x\in L \iff \exists y_1 \; s.t.\; (x,y_1) \in L' \iff \\
\exists y_1,y_2  \forall y_3\ldots Q_ky_k \;M'(x,y_1,y_2,\ldots,y_k) =1 
\end{align*}
And we got that $L\in \Sigma_{k-1}^p$, which by IA we know $\Sigma_{k-1}^p\subseteq NP$, and we derive our desire conclusion:
\begin{align*}
L\in NP \\
\end{align*}
And thus,
\begin{align*}
\Sigma_k^p \subseteq NP
\end{align*}
By induction we showed that for any $i\in \mathbb{N}$, $\Sigma_i^p\subseteq NP$ and thus $PH\subseteq NP$, we know $NP\subseteq PH$, so we conclude: $PH= NP$.
\section*{Question 3}
\subsection*{a. }
Assume $DP\subseteq NP$, for any $L\in CO-NP$, if we will look at: \begin{align*}
	L\cap \Sigma^* = L
\end{align*}
Because $\Sigma^*\in P \subseteq NP$ we got $L\in DP$. So we showed that also:\
\begin{align*}
CO-NP \subseteq DP
\end{align*}
Together with our assumption we got:
\begin{align*}
CO-NP \subseteq DP \subseteq NP
\end{align*}
And we already showed in Q.2 that $CO-NP \subseteq NP$ is concluding to $PH=NP$.
\subsection*{b. }
First I will show $EC$ (Exact -Clique) is in $DP$, as we know $CLIQUE\in NP$
\begin{align*}
CLIQUE = \{ <G,k> : \text{there exists clique in } G \text{ of size } k \}
\end{align*}
So for: 
\begin{align*}
CLIQUE_{+1} = \{ <G,k> : \text{there exists clique in } G \text{ of size } k+1 \}
\end{align*}
Clearly $CLIQUE_{+1} \in NP$, so $\overline{CLIQUE_{+1}} \in CO-NP$ and we got:
\begin{align*}
EC = CLIQUE \cap \overline{CLIQUE_{+1}}
\end{align*}
And we conclude $EC\in DP$.\\
To show $EC \in DP-hard$, given $L\in DP$ I will show how to reduce it in polynomial time to $EC$. I will use the known fact $CLIQUE\in NP-Complete$.
From this fact we know that $\overline{CLIQUE}\in CO-NP-Complete$.
Given $L\in DP$, we know:
\begin{align*}
L = L_1\cap L_2\quad  where:\\
L_1\in NP\quad L_2\in CO-NP
\end{align*}
And because $CLIQUE\in NP-Complete$ and $\overline{CLIQUE}\in CO-NP-Complete$, there those reductions:
\begin{align*}
L_1 \le_p CLIQUE \\
L_2 \le_p \overline{CLIQUE}
\end{align*}
So there exists two polynomial reduction function $f_1,f_2$ s.t.:
\begin{align*}
x\in L_1 \iff f_1(x)=<G_1,k_1> \in CLIQUE \\
x \in L_2 \iff f_2(x)=<G_2,k_2> \in \overline{CLIQUE}
\end{align*}
\textbf{Notice: } WLOG we can assume that $k_1 \ne k_2$ as if it's not the case there $f'_1$ which will just add another node to $G_1$, the output of $f_1(x)$ and connect it to all the nodes, and set $k'_1 = k_1 +1$. $f'_1$ will be also a reduction function with the same properties and will have $k'_1 \ne k_2$.\\ \\
Recall that the standard reduction construction from $3-SAT$ to $CLIQUE$ are having that if there is a clique in size $k_i$ this is the biggest clique. (As $k_i$ is the biggest size of clique possible, if by contradiction there exists a clique of size bigger than $k_i$ it will lead by the pigeonhole principle to two nodes from the same clause connected together). So we can adjust the previous equalities:
\begin{align*}
&x\in L_1 \iff f_1(x)=<G_1,k_1> \text{and the biggest clique in }G_1\text{ is in size }k_1. \\
&x \in L_2 \iff f_2(x)=<G_2,k_2> \text{and the biggest clique in } G_2 \text{ is in size smaller than } k_2
\end{align*}
We will set new reduction function $f'_i$ ($i\in \{1,2 \}$) on input $x$, $f'_i$ will run $f_i(x)$ and receive $<G_i,k_i>$ it will add $K_{{k_i}-1}$ to $G_i$ denote it by $G'_i$ and return $<G'_i,k_i>$, where $K_j$ is a clique in size $j$. Now we will have those properties: 
\begin{align*}
&x\in L_1 \Rightarrow  f'_1(x)=<G'_1,k_1> \text{and the biggest clique in }G'_1\text{ is in size }k_1\\
&x\notin L_1 \Rightarrow  f'_1(x)=<G'_1,k_1> \text{and the biggest clique in }G'_1\text{ is in size }k_1-1 \\
&x \in L_2 \Rightarrow  f'_2(x)=<G'_2,k_2> \text{and the biggest clique in }G'_2\text{ is in size }k_2-1 \\
&x\notin L_2 \Rightarrow   f'_2(x)=<G'_2,k_2> \text{and the biggest clique in }G'_2\text{ is in size }k_2  \\
\end{align*}
Now we ready to set our reduction function $f$ from $L$ to $EC$. Given $x\Sigma^*$ we will run $f'_1(x),f'_2(x)$ and get $<G'_1,k_1>$ and $<G'_2,k_2>$ where by our notice $k_1\ne k_2$, we return $f(x) =<G'_1\times G'_2,k_1\cdot (k_2-1)>$ ($G'_1\times G'_2$ will just return both union of the graphs with addition than each node in $G'_1$ will be connected to all the  nodes of $G'_2$). And we got:
\begin{align*}
&x\in L \Rightarrow x\in L_1 \wedge x \in L_2 \Rightarrow \\
& \text{the biggest clique in }G'_1\text{ is in size }k_1 \wedge \\
&\text{the biggest clique in }G'_2\text{ is in size }k_2-1 \Rightarrow\\
& \text{the biggest clique in  } G'_1\times G'_2  \text{ is of size } k_1\cdot (k_2-1) \Rightarrow\\
& f(x)\in EC\\
\\
&x\notin L \Rightarrow x\notin L_1 \vee x \notin L_2 \Rightarrow \\
& \text{the biggest clique in }G'_1\text{ is in size }k_1-1 \vee \\
&\text{the biggest clique in }G'_2\text{ is in size }k_2 \Rightarrow\\
& \text{the biggest clique in  } G'_1\times G'_2  \text{ is size } ((k_1-1)\cdot k_2) \text{ or } (k_1\cdot k_2) \text{ or } ((k_1-1)\cdot (k_2-1)) \\
&\text{which in any case } \ne (k_1\cdot (k_2-1)) \Rightarrow\\
& f(x)\notin EC
\end{align*}
\newpage
\section*{Question 4}
\subsection*{a. }
We will show algorithm that uses only logarithmic space and will decide $FVAL$. Given $(F,a)$, for any literal in $F$ we refer in our algorithm $a(n)\in \{T,F\}$ as the substition of $a$ on that literal, notice that this can be checked with $O(1)$ space. In our algorithm we will think of $F$ as a binary tree, and use the fact that we can decide for any node $n$ in tree $F$ which direct subtree of $n$ is bigger in logarithmic space. This can be calling the following sub routine twice once where $n$ is the left subtree and after setting $n$ to be the right subtree.\\
\begin{algorithm}
	\begin{algorithmic}[1]
		\Procedure{CheckSize}{}
		\State $counter \gets 0$
		\State $level \gets 0$
		\State $cameFrom \gets \emptyset$
		\BState \emph{loop}:
		\If {$n \text{ is litreal} $ \textbf{or} $cameFrom = 'right'$} 
			\State $counter \gets counter +1$
			\If {$level = 0$} 
			\State \textbf{return} counter
			\Else
			\State $level \gets level-1$
			\If{$n \text{ is a right node}$}
			\State	$cameFrom \gets  'right'$
			\Else
			\State	$cameFrom \gets  'left'$
			\EndIf
			\State $n \gets  n.parent$
			\State 	\textbf{goto} loop
			\EndIf
		\ElsIf {$cameFrom = 'left'$}
			\State $level \gets level+1$
			 \State $n \gets  n.rightnode$
			 \State $cameFrom \gets \emptyset$
			 \State 	\textbf{goto} loop
		\ElsIf {$cameFrom = \emptyset$}
		\State $level \gets level+1$
		\State $n \gets  n.leftnode$
		\State 	\textbf{goto} loop
		
		\EndIf
		\EndProcedure
	\end{algorithmic}
\end{algorithm}

Than we will compare between those counters and decide which subtree is bigger. We used here 4 variables which at most need $log(N)$ space where $N$ is number of nodes in tree which is $O(n)$. We used here and in our algorithm the notation $n$ for variable for node, we can think of $n$ denote the subtree we are looking now as two pointers that mark the position in $F$, remark that each pointer doesn't need more than $log(n)$ space. We can update $n\leftarrow n.parent$ by just update the pointers to the outside parentheses. \\
Now we are ready to present the algorithm that will decide $FVAL$, this algorithm will basically will valuate the tree $F$ from down to up traversing through the bigger subtree in each step:
\begin{algorithm}
	\caption{FVAL(F,a)}\label{euclid}
	\begin{algorithmic}[1]
		\State $ans,cameFrom,S \gets \emptyset$
		\State $i,level \gets 0$
		\State $n \gets \text{root of F}$
		\BState \emph{loop}:
		\If {$n \text{ is litreal} $} 
			\State $ans \gets a[n]$
			\BState \emph{levelup}:
			\If {$level = 0$}  \textbf{return} $ans$
			\ElsIf {$n \text{ is a right node}$}
				\State	$cameFrom \gets  'right'$
			\Else
				\State	$cameFrom \gets  'left'$
			\EndIf
			\State $n \gets  n.parent$
			\State 	\textbf{goto} loop
		\ElsIf {$ans = \emptyset$}
			\If {$(CheckSize(n.rightnode) > CheckSize(n.leftnode) )$} $n \gets n.rightnode$
			\Else $ \quad n \gets   n.leftnode$
			\EndIf
			\State $level \gets level+1$
			\If {$S[i] \ne null $} $i \gets i+1$
			\EndIf
			\State 	\textbf{goto} loop
		\ElsIf {$S[i] = null $}
			\State $S[i] \gets ans$
			\State $ans \gets  \emptyset$
			\If {$cameFrom = 'right'$}
				\State $n \gets  n.leftnode$
			\Else
				\State $n \gets  n.rightnode$
			
			\EndIf
			\State $cameFrom \gets \emptyset$
			\State 	\textbf{goto} loop
		\Else
			\If {n is from the $F_1\wedge F_2$ format} 
				\State $ans \gets min\{ ans, S[i] \}$
			\Else 
				\State $ans \gets max\{ ans, S[i] \}$
			\EndIf
			\State $S[i] \gets null $
			\State $level \gets level-1$
			\If {$i>level$} $i \gets level$
			\EndIf
			\State 	\textbf{goto} levelup
		\EndIf
	\end{algorithmic}
\end{algorithm}

\newpage
$ans,cameFrom$ takes only $O(1)$ space, $i,level$ are representing a number which is big at most as $n$, so we need only $O(log(n))$ space to save them. For $n$ we already explained previously how can we managed it with only $O(log(n))$ bits. So the only thing we need to ensure is that the array $S$ is not taking more than $O(log(n))$ space:
\begin{lemma} through all the run of the algorithm $S$ use bits at most as the depth of the biggest full subtree of $F$ +1 .
\end{lemma}
By the lemma we conclude $S$ is using at most $log(n)$ bits, as the size of the biggest full subtree of $F$ is smaller than $n$ and it's depth is $log(n)$.
\\
\\ \textbf{Proof of lemma} by induction on the size of the subtree the algorithm check in the $loop$. If subtree is in size of one (literal) we don't use $S$. Assume that for any subtree $F'$ we checked in $loop$ of size $k,k <n$ the if $F'$ was full subtree then $S$ grew at most to the size of $F'$'s height and otherwise it grew to size of the biggest full subtree of $F'$ +1 . Let be $F'$ be a subtree of size $n$. If $F'$ is full tree with length $l$, so the calculation on it's left subtree (which is also full subtree) will take at most $l-1$ bits in $S$ (by induction assumption), after we go back to $F'$ and before checking the right subtree, $S$ will shrink to only one value. Then we will use additional at most $l-1$ bits for the right subtree of $F'$, by the induction assumption, with total of $l$ bits. Otherwise $F'$ is not a full tree denote by $l$ the height of the biggest full subtree of $F'$.
To calculate the first subtree of $F'$ by IA we will at most $l+1$ bits in $S$, but after coming back to $F'$, $S'$ will shrink back to size 1. It could not be the case that the biggest full subtree of $F'$ is direct child of it in the second subtree calculation, as we scan $F'$ first by the biggest child and this will lead to the fact that $F'$ is full by it own, which is not the case we are in. Therefore to calculate the second subtree we will need, by IA, $l+1$ which will shrink to size two after coming back to second subtree of $F'$ (one saved for the first subtree and the second bit for the second subtree). Then we will go back to $F'$ and finish.
\subsection*{b. } $SAT$ is a special case of boolean formula, so as we showed in 1, iff $x\in SAT$ there is a witness for it, a interpretation for it $a$ with:  
\begin{align*}
(x,a)\in FVAL
\end{align*}
Because the witness tape is just like a normal tape we can run the algorithm from section a, and determine $x\in SAT$ iff $\exists a$ s.t. $ (x,a)\in FVAL$. 
\subsection*{c. }
\begin{claim}
$G$ is not bipartite graph $\iff$ there is a cycle of odd length in $G$
\end{claim}
We will show that $BIPARTITE \in CO-NL$ and by theorem we showed in class ($NL=CO-NL$) we will conclude $BIPARTITE\in NL$. Let us show $\overline{BIPARTITE}\in NL$ by describe $M$ non deterministic TM which decide $\overline{BIPARTITE}$.\\
$M$ will receive as an witness the odd cycle in $G$ (by our claim we know there exist iff $G\in \overline{BIPARTITE}$) and just check that this cycle exists in $G$.\\
$M$ will accept $\iff$ there is such witness $\iff$ $G \in \overline{BIPARTITE}$.
\begin{claimproof}
($\Leftarrow$) Let $C = \{ v_1,v_2,\ldots,v_{2n+1} \}$ be a cycle of odd length, by contradiction assume $G=(V\cup E)$ is bipartite with $V=L\cup R$ division of $V$ to two disjoint sets as promised. WLOG assume $v_1\in L$ so $v_2\in R$, $v_3\in L$ and so on (there is edge between those vertices so they have to be in different sets), we conclude all the odd vertices are in $L$ but there is edge between $v_{2n+1}$ to $v_1$ and we arrive to contradiction.\\
($\Rightarrow$) Let be 
\end{claimproof}	
\section*{Question 5}
\subsection*{a. }
Let be $f$ a one-directional function, and we will show $P=NP$.\\
Let $M_f$ be the polynomial TM which compute $f$.\\
Let us define TM $M(y,x)$, 
\begin{itemize}
\item given $(y,x)$
\item $M$ runs $M_f(x)$ and get $y'$
\item $M$ accepts iff $y'=y$
\end{itemize}
So $M$ is polynomial. We will define now $L'$:
\begin{align*}
L' = \{ (y,x,1^n) : \exists z \;s.t\quad M(y,x\cdot z)=1 \wedge |x\cdot z| =n \}
\end{align*}
We will show $L'\in NP\setminus P$:
\begin{itemize}
	\item $L'\in NP$: We will show a non deterministic TM $M'$ which will decide $L'$. Given $(y,x,1^n)$, $M'$ will will guess $z\in \{0,1\}^{n-|x|}$ and run $M(y,x\cdot z)$ and answer the same. We will have:
	\begin{align*}
	(y,x,1^n) \in L' \iff \exists z \;s.t\quad M(y,x\cdot z)=1 \wedge |x\cdot z| =n \iff \\ there \; exists\;accepting\; computation\; for\;M'
	\end{align*}
	\item $L'\notin P$: Otherwise by contradiction $L'\in P$, there exists TM $M_p$ s.t 
	\begin{align*}
	(y,x,1^n) \in L' \iff  M_p(y,x,1^n)=1 \iff  \exists z \;s.t\quad M(y,x\cdot z)=1 \wedge |x\cdot z| =n 
	\end{align*}
	So we can build the following algorithm $\mathcal{A}$ which reverse $f$, like that, given $y$ and $1^{|x|}$, $\mathcal{A}$ will set the first bit of $x$ to 1 if the run of $M_p(y,1,1^{|x|})$ accepts, and 0 otherwise. Continue by running $M_p(y,x_11,1^{|x|})$ to determine the second bit of $x$ and after total $|x|$ runs of $M_p$ we will find $x$, and hence $\mathcal{A}$ is polynomial in $(y,1^{|x|})$. We got:
		\begin{align*}
	\Pr_{x^R\in\{0,1\}^*,f(x)=y} [ \mathcal{A}(y)=x' \quad s.t.\quad f(x')=y]=1 >\frac{1}{n}
	\end{align*}
	for all $n>1$, and contradiction that $f$ is one-directional function.
\end{itemize}



\subsection*{b. } Assume $F$ is running in $n^c$, $c>2$, for input in size $n$, we will define new function $F'$ which will run in time $n^2$. Given $x$, $F'$ will run $F$ on the first $|x|^{\frac{2}{c}}$ bits of $x$ (deleting all the bits after those first bits) and return. Clearly the run time of $F'$ is ${n^{\frac{2}{c}}}^c = n^2$. If by contradiction $F'$ is not one-directional, there is algorithm $\mathcal{A}$ s.t. 
	\begin{align}\label{pr}
\Pr_{x^R\in\{0,1\}^*,f(x)=y} [ \mathcal{A}(y)=x' \quad s.t.\quad F'(x')=y]>\epsilon(n) 
\end{align}
For any negligible function $\epsilon(n)$. So we can describe new algorithm $\mathcal{A}'$ which will contradict the face $F$ is one-directional. Given $(y, 1^{|x|})$, for some $x\in \{0,1\}^*$ which $F(x)=y$, $\mathcal{A}'$ will run $\mathcal{A}$ on $(y,1^{|x|^{\frac{c}{2}}}) $ which will return $x'$ s.t $F'(x') = y$ with high probability [as described in \ref{pr}], $\mathcal{A}'$ return the first $|x'|^{\frac{2}{c}}$ of $x'$, denote the output by $x''$, the size of output will be:
\begin{align*}
|x''| = |x'| ^ {\frac{2}{c}} = {|x|^{\frac{c}{2}}}^{\frac{2}{c}} = |x|
\end{align*}
By the definition of $F'$, $x''$, the first $|x'|^{\frac{2}{c}}$ bits of $x'$, are fulfilling $F(x'') = y$, and we arrive to contradiction that $F$ is one directional.
\end{document}
